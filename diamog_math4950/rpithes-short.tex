%%%%%%%%%%%%%%%%%%%%%%%%%%%%%%%%%%%%%%%%%%%%%%%%%%%%%%%%%%%%%%%%%%% 
%                       rpithes-short.tex                         %
%         Template for a short thesis all in one file             %
%        (titlepage info below assumes masters degree}            %
%  Just run latex (or pdflatex) on this file to see how it looks  %
%      Be sure to run twice to get correct TOC and citations      %
%%%%%%%%%%%%%%%%%%%%%%%%%%%%%%%%%%%%%%%%%%%%%%%%%%%%%%%%%%%%%%%%%%% 
%
%  To produce the abstract title page followed by the abstract,
%  see the template file, "abstitle-mas.tex"
%
%%%%%%%%%%%%%%%%%%%%%%%%%%%%%%%%%%%%%%%%%%%%%%%%%%%%%%%%%%%%%%%%%%%

\documentclass{thesis}
\usepackage{graphicx}   % if you want to include graphics files

% Use the first command below if you want captions over 1 line indented.
% A side effect of this is to remove the use of bold for captions. 
% To restore bold, also include the second line below.
%\usepackage[hang]{caption}     % to indent subsequent lines of captions
%\renewcommand{\captionfont}{\bfseries} % only needed with caption package;
                                        %   otherwise bold is default)
                                        
%%%%%%%%%%%%%%%%%%%%  supply titlepage info  %%%%%%%%%%%%%%%%%%%%%
\thesistitle{\bf Diffusive Load Balancing\\of Shape and Neighbor Counts}        
\author{Gerrett Diamond, Cameron Smith}        
\degree{Bachelors of Science}
\department{Mathematics} % provide your area of study here; e.g.,
%  "Mechanical Engineering", "Nuclear Engineering", "Physics", etc.
\thadviser{Mark Shephard}
%\cothadviser{First co-adviser} %if needed
%\cocothadviser{Second co-adviser} % if needed
%  For a masters project use \projadviser instead of \thadviser, 
%  and \coprojadviser and \cocoprojadviser if needed. 
\submitdate{December 2014\\(For Graduation May 2015)}        
%\copyrightyear{1685}  % if date omitted, current year is used. 
%%%%%%%%%%%%%%%%%%%%%   end titlepage info  %%%%%%%%%%%%%%%%%%%%%%
      
\begin{document} 
\titlepage             % Print titlepage   
%\copyrightpage        % optional         
\tableofcontents       % required 
\listoftables          % required if there are tables
\listoffigures         % required if there are figures

\specialhead{ACKNOWLEDGMENT}
The acknowledgment text goes here. Unlike chapter headings, 
this heading is not numbered.

\specialhead{ABSTRACT}
Massively parallelized mesh simulations require the mesh to be distributed
throughout many processes(parts) and be balanced in terms of computation and
communicaion. 
Many algorithms exist that target having an equal number mesh entities of a
specified dimension on each part. 
The most powerful of these, in terms of partition quality and runtime are the
multi-level (hyper)graph-based methods and the recusive geometric sectioning
methods. 
Multi-level (hyper)graph methods describe the mesh in terms of graph nodes and
(hyper)edges and then use a v-cycle of coarsening followed by un-coarsening to
divide the graph into subgraphs while recursive sectioning geometric methods use
a coordinate system such as the cartesian location or centroids to recursivly
section the mesh along coordinate or inertial axis. 
The faster geometric methods tend to create parts that have large inter-part
surface area, and many neighbors relative to the (hyper)graph based methods
which causes an increase in communication.
(Hyper)Graph-based methods can also suffer from similar issues as the number of
parts gets very large and the number of elements per part drops to several
hundreds. 
An alternative method that uses mesh data directly, ParMA, performs load 
balancing without the need to construct a graph. 
With access to all the mesh, partition, and model data, including topological 
and geometric coordinates, partitioning methods analogous to those used by 
(hyper)graph and geometric based procedures can be defined.
Towards this, ParMA paritioning methods will be extended to improve the overall 
shape of a part in order to reduce the number of neighbors each part has and 
thus reduce application communication times.

\chapter{INTRODUCTION}
The text of the first chapter goes here. To cite a reference for the
bibliography, use a command such as:\cite{thisbook}
\section{Motivation}
This is a sentence to take up space and look like text.
%describe terminology (heavy vs light parts)

\chapter{Related Works}

Many different distributed methods to balance unstructured meshes have been explored in the 
past. Here we discuss the three most commonly used methods: diffusive, hypergraph, 
and multilevel.

\section{Diffusive load balancing}
Willebeek-LeMair and Reeves describe diffusive load balancing techniques with 
4 steps \cite{loadbalance}. The first is evaluating the processor load, then 
determining the profitability of the load balancing, followed by a task 
migration strategy, and finally a task selection strategy. They review over 5 
different strategies and how they perform on the second and thrid task. The first 
method reviewed was the Sender Initiated Diffusion method (SID). This method 
restricts parts to only communicate with their neighbors and the heavy parts or 
senders are the migration selectors. The second technique is similar to the first 
and called Reciever Initiated Diffusion (RID). Once again parts can only 
communicate with their neighbors, but instead of the heavy parts selecting 
migration targets, the light parts select heavier neighbors to recieve load from. 
Our method for shape optimization is similar to the idea of a RID method. In the 
Gradient Method (GM), light parts inform nearby parts in search of a heavy part 
which causes heavy parts to send load towards the closest light part. The 
Hierarchial Balancing Method (HBM) breaks up load balancing into a hierarchy of 
parts and performs load balancing at a low level and then continues to balance 
while moving up a hierarchy. The final method is the Dimmension Exchange Method
 (DEM). Similar to the HBM, the DEM approach balances at a low level and works 
up, but the DEM balances synchronously in all dimensions one at a time.

\section{Hypergraph methods}

\section{Multilevel methods}

Another citation for the bibliography:\cite{anotherbook}


\chapter{Methods}

\section {Load Evaluation}

\section{Task migration strategy}

\section{Task selection strategy}


\chapter{Implementation}
% The following produces a numbered bibliography where the numbers
% correspond to the \cite commands in the text.
\specialhead{LITERATURE CITED}
\begin{singlespace}
\begin{thebibliography}{99}
\bibitem{multidiffuse} Meyerhenke, H.; Monien, B.; Sauerwald, T., "A new diffusion-based multilevel algorithm for computing graph partitions of very high quality," Parallel and Distributed Processing, 2008. IPDPS 2008. IEEE International Symposium on , vol., no., pp.1,13, 14-18 April 2008
\bibitem{diffuseshape} Ralf Diekmann, Robert Preis, Frank Schlimbach, Chris Walshaw, Shape-optimized mesh partitioning and load balancing for parallel adaptive FEM, Parallel Computing, Volume 26, Issue 12, November 2000, Pages 1555-1581, ISSN 0167-8191, http://dx.doi.org/10.1016/S0167-8191(00)00043-0.
\bibitem{loadbalance} Willebeek-LeMair, M.H.; Reeves, AP., "Strategies for dynamic load balancing on highly parallel computers," Parallel and Distributed Systems, IEEE Transactions on , vol.4, no.9, pp.979,993, Sep 1993
\bibitem{thisbook} This is the first item in the Bibliography.
Let's make it very long so it takes more than one line.
Let's make it very long so it takes more than one line.
\bibitem{anotherbook} The second item in the Bibliography.
\end{thebibliography}
\end{singlespace}

%%%%%%%%%%%%%%%%%%%%%%%  For Appendices  %%%%%%%%%%%%%%%%%%%
\appendix    % This command is used only once!
\addtocontents{toc}{\parindent0pt\vskip12pt APPENDICES} %toc entry, no page #
\chapter{THIS IS AN APPENDIX}
Note the numbering of the chapter heading is changed.
This is a sentence to take up space and look like text.
\section{A Section Heading}
This is how equations are numbered in an appendix:
\begin{equation}
x^2 + y^2 = z^2
\end{equation} 

\chapter{THIS IS ANOTHER APPENDIX}
This is a sentence to take up space and look like text.

\end{document}
