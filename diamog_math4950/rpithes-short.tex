%%%%%%%%%%%%%%%%%%%%%%%%%%%%%%%%%%%%%%%%%%%%%%%%%%%%%%%%%%%%%%%%%%% 
%                       rpithes-short.tex                         %
%         Template for a short thesis all in one file             %
%        (titlepage info below assumes masters degree}            %
%  Just run latex (or pdflatex) on this file to see how it looks  %
%      Be sure to run twice to get correct TOC and citations      %
%%%%%%%%%%%%%%%%%%%%%%%%%%%%%%%%%%%%%%%%%%%%%%%%%%%%%%%%%%%%%%%%%%% 
%
%  To produce the abstract title page followed by the abstract,
%  see the template file, "abstitle-mas.tex"
%
%%%%%%%%%%%%%%%%%%%%%%%%%%%%%%%%%%%%%%%%%%%%%%%%%%%%%%%%%%%%%%%%%%%

\documentclass{thesis}
\usepackage{graphicx}   % if you want to include graphics files

% Use the first command below if you want captions over 1 line indented.
% A side effect of this is to remove the use of bold for captions. 
% To restore bold, also include the second line below.
%\usepackage[hang]{caption}     % to indent subsequent lines of captions
%\renewcommand{\captionfont}{\bfseries} % only needed with caption package;
                                        %   otherwise bold is default)
                                        
%%%%%%%%%%%%%%%%%%%%  supply titlepage info  %%%%%%%%%%%%%%%%%%%%%
\thesistitle{\bf Diffusive Load Balancing\\of Shape and Neighbor Counts}        
\author{Gerrett Diamond, Cameron Smith}        
\degree{Bachelors of Science}
\department{Mathematics} % provide your area of study here; e.g.,
%  "Mechanical Engineering", "Nuclear Engineering", "Physics", etc.
\thadviser{Mark Shephard}
%\cothadviser{First co-adviser} %if needed
%\cocothadviser{Second co-adviser} % if needed
%  For a masters project use \projadviser instead of \thadviser, 
%  and \coprojadviser and \cocoprojadviser if needed. 
\submitdate{March 2015\\(For Graduation May 2015)}        
%\copyrightyear{1685}  % if date omitted, current year is used. 
%%%%%%%%%%%%%%%%%%%%%   end titlepage info  %%%%%%%%%%%%%%%%%%%%%%
      
\begin{document} 
\titlepage             % Print titlepage   
%\copyrightpage        % optional         
\tableofcontents       % required 
\listoftables          % required if there are tables
\listoffigures         % required if there are figures

\specialhead{ACKNOWLEDGMENT}
The acknowledgment text goes here. Unlike chapter headings, 
this heading is not numbered.

\specialhead{ABSTRACT}
Massively parallel mesh simulations require the mesh to be distributed
throughout many processes(parts) and be balanced in terms of computation and
communication. 
Many algorithms exist that target having an equal number mesh entities of a
specified dimension on each part. 
The most powerful of these, in terms of partition quality and run time are the
malt's-level (hyper)graph-based methods and the recursive geometric sectioning
methods. 
Multi-level (hyper)graph methods describe the mesh in terms of graph nodes and
(hyper)edges and then use a v-cycle of coarsening followed by un-coarsening to
divide the graph into subgraphs while recursive sectioning geometric methods use
a coordinate system such as the Cartesian location or centroids to recursively
section the mesh along coordinate or inertial axis. 
The faster geometric methods tend to create parts that have large inter-part
surface area, and many neighbors relative to the (hyper)graph based methods
which causes an increase in communication.
(Hyper)Graph-based methods can also suffer from similar issues as the number of
parts gets very large and the number of elements per part drops to several
hundreds. 
An alternative method that uses mesh data directly, ParMA, performs load 
balancing without the need to construct a graph. 
With access to all the mesh, partition, and model data, including topological 
and geometric coordinates, partitioning methods analogous to those used by 
(hyper)graph and geometric based procedures can be defined.
Towards this, ParMA partitioning methods will be extended to improve the 
overall shape of a part in order to reduce the number of neighbors each part 
has and thus reduce application communication times.

\chapter{INTRODUCTION}
The text of the first chapter goes here. 

%describe terminology (heavy vs light parts)
\section{Definition of mesh and graph terminology} 

\section{Motivation}
In various applications it is important to reduce the average number of 
neighbors a part has. If parts share boundaries with many neighbors there will 
be lots of time spent sending and receiving information from each part's 
neighbors.

\chapter{Related Works}
Many different distributed methods to balance unstructured meshes have been 
explored in the past \cite{multidiffuse,surveygraph}. Here we discuss the 
three most commonly used methods: graph/hypergraph, diffusive, and multilevel.

\section{(Hyper)graph methods}
Graph and Hypergraph load balancing methods use an abstraction of the mesh in 
order to balance the work load between processes. The first step to using 
hypergraph/graph methods is creating this abstraction. Zhou et al. define that 
the graph nodes are built off of a specific mesh entity typically either faces 
for 2D or regions for 3D \cite{zhougraph}. The edges are then built based on the
dependencies of the nodes. In their work they found that using only a subset of 
these edges is sufficient for the same quality of partition. For their work with
regions as nodes they found using the face adjancencies gave just as good a 
partition in much less time. The authors also define two different types of 
edges: Interedges that connect graph nodes on different processors and 
intraedges that connect graph nodes on the same processor. Zhou describes the 
hypergraph as an extension of the graph where hypergraph nodes are the same as 
graph nodes, but hypergraph edges or hyperedges represent dependencies between 
more than one mesh entity. Thus hyperedges can correlate each entity with all 
of its face adjancencies. The overall goal of hypergraphs and hyperedges is to 
partition the graph while minimizing the number of parts the hyperedges extend 
over.

Buluc et al. \cite{surveygraph} give an overview of graph methods in the field 
for both graphs and hypergraphs. They also discuss the challenges and 
NP-hardness of many of the algorithms for graph balancing. Two metrics that 
are used to model the overall shape for graph methods are the edge cut and 
communication volume. The authors show that communication volume is a better 
measurement which leads to hypergraph methods which better target the 
communication volume because hyperedges represent the number of different 
neighboring parts to each node in the graph. Devine et al. discuss methods 
to use hypergraphs for parallel scientific computing \cite{hypergraph}. They
use hypergraphs to accurately represent communication volume and as a result
part neighbors. The hypergraph method uses hyperedges or nets which connects 
several nodes in the graph that may be on different parts. Therefore 
balancing a hypergraph naturally puts an effort towards balancing the overall
shape and neighbors of parts. 

Graph and Hypergraph methods for load balancing are said to be the most 
effective \cite{zhougraph,surveygraph}. Still there are some draw backs 
presented by Zhou et al. \cite{zhougraph}. First off since the graph targets 
a specific dimension, the other dimensions can become unbalanced. Also 
graph/hypergraph methods do not scale well when working with more than 
100,000 processors. The authors explain that using both local partitioning 
and global partitioning can result in good partitions that can scale out to
large part counts. Also with large part counts some parts become much 
heavier than others which limits the scalability of these methods.

\section{Diffusive load balancing}
Willebeek-LeMair and Reeves describe diffusive load balancing techniques with 
four steps \cite{loadbalance}. The first is evaluating the processor load, then 
determining the profitability of the load balancing, followed by a task 
migration strategy, and finally a task selection strategy. They review over 
five different strategies and how they perform on the second and third task. 
The first method reviewed was the Sender Initiated Diffusion method (SID). 
This method restricts parts to only communicate with their neighbors and the 
heavy parts or senders are the migration selectors. The second technique is 
similar to the first and called Receiver Initiated Diffusion (RID). Once again 
parts can only communicate with their neighbors, but instead of the heavy parts
selecting migration targets, the light parts select heavier neighbors to receive
load from. Our method for shape optimization is similar to the idea of a RID 
method. In the Gradient Method (GM), light parts inform nearby parts in search 
of a heavy part which causes heavy parts to send load towards the closest light 
part. The Hierarchical Balancing Method (HBM) breaks up load balancing into a 
hierarchy of parts and performs load balancing at a low level and then continues
to balance while moving up a hierarchy. The final method is the Dimension 
Exchange Method (DEM). Similar to the HBM, the DEM approach balances at a low 
level and works up, but the DEM balances synchronously in all dimensions one at 
a time.

Zhou et al. developed a diffusive algorithm to balance the heavy parts produced 
by graph/hypergraph methods \cite{zhougraph}. They target balancing the mesh vertices for each
part of the partition by offloading certain entities from heavy parts to 
lighter parts. Their method, the local iterative interpart boundary 
modification algorithm (LIIPBMod), identifies mesh vertices on the heavy parts
that are bounded by a small number of elements. By doing this, LIIPBMod 
balances the vertex partitioning while still maintaining the element balance
produced from the graph/hypergraph partition. This process when repeated 
results in a partition that has vertex balance with a minor decrease in the 
good element balance. The results of their method shows a dramatic improvement
of vertex imbalance sometimes reducing from over 20\% to around 5\%. They 
discuss that the scaling is not so significant because much more time is spent 
on the graph/hypergraph partitioning than their method.


\section{Multilevel methods}
Buluc et al. describe the multilevel methods as the most successful approach to 
graph partitioning \cite{surveygraph}. They describe multilevel methods 
generally as a three step process. The first step is to coarsen the mesh 
until some threshold in order to reduce complexity. The second is to create 
an initial partition of this fully coarsened mesh that can be done much 
faster than the entire mesh. Next the mesh is refined in the opposite direction
of the coarsening. At each step the mesh is progressively refined in order to
keep the balance across the parts. This design is called a v-cycle. The 
generality of the design allows most algorithms to be altered to work in a 
multilevel form. 

Meyerhenke et al. port a diffusive method previously used for 
shape optimization into a multilevel algorithm \cite{multidiffuse}. They use a 
previously used distributed load balancing algorithm, Bubble-FOS/C (First order
diffusion scheme with constant drain) for load balancing in their multilevel
method. This Bubble method is an iterative scheme based on Lloyd's k-means 
algorithm. This method involves solving linear systems repeatedly which causes 
a high run time. Thus the authors introduce a second balancing algorithm, 
TruncCons (truncated diffusion consolidations), into the v-cycle. This 
algorithm is faster than Bubble for large mesh sizes and is used for small
diffusive improvements. They use the multilevel structure in order to use 
the expensive but more effective Bubble-FOS/C when at the small levels of the 
v-cycle and then use TruncCons to balance the larger parts of the v-cycle. 
This method was shown to be much faster than the original method 
of just the Bubble-FOS/C and also produces a better quality of load balancing. 
They compared this new method to the popular methods KMetis and JOSTLE and 
found their method to produce higher quality partitions but still having a
much higher run time. They expect reasonable scaling for the algorithm but 
results show the algorithm taking minutes rather than seconds like other 
methods.

\chapter{GAP}

\chapter{Limitations}

\chapter{Results}

\chapter{Future Work}

\chapter{Conclusion}

% The following produces a numbered bibliography where the numbers
% correspond to the \cite commands in the text.
\specialhead{LITERATURE CITED}
\begin{singlespace}
\begin{thebibliography}{99}
\bibitem{multidiffuse} Meyerhenke, H.; Monien, B.; Sauerwald, T., "A new 
diffusion-based multilevel algorithm for computing graph partitions of very 
high quality," Parallel and Distributed Processing, 2008. IPDPS 2008. IEEE 
International Symposium on , vol., no., pp.1,13, 14-18 April 2008
\bibitem{diffuseshape} Ralf Diekmann, Robert Preis, Frank Schlimbach, Chris 
Walshaw, Shape-optimized mesh partitioning and load balancing for parallel 
adaptive FEM, Parallel Computing, Volume 26, Issue 12, November 2000, Pages 
1555-1581, ISSN 0167-8191, http://dx.doi.org/10.1016/S0167-8191(00)00043-0.
\bibitem{loadbalance} Willebeek-LeMair, M.H.; Reeves, AP., "Strategies for 
dynamic load balancing on highly parallel computers," Parallel and Distributed 
Systems, IEEE Transactions on , vol.4, no.9, pp.979,993, Sep 1993
\bibitem{zhougraph} Zhou M, Sahni O, Devine KD, Shephard MS, Jansen KE (2010) 
Controlling unstructured mesh partitions for massively parallel simulations. 
SIAM J Sci Comput 32:3201–3227
\bibitem{surveygraph} A. Buluc, H. Meyerhenke, I. Safro, P. Sanders, and C. 
Schulz, Recent advances in graph partitioning , (2013), p. arXiv:1311.3144.
\bibitem{hypergraph}  Karen D. Devine , Erik G. Boman , Robert T. Heaphy , 
Rob H. Bisseling , Umit V. Catalyurek, Parallel hypergraph partitioning for 
scientific computing, Proceedings of the 20th international conference on 
Parallel and distributed processing, p.124-124, April 25-29, 2006, Rhodes 
Island, Greece 
\end{thebibliography}
\end{singlespace}

\end{document}
