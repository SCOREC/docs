\documentclass[a4paper]{article}

\usepackage[english]{babel}
\usepackage[utf8x]{inputenc}
\usepackage{amsmath}
\usepackage{graphicx}
\usepackage{algorithm}
\usepackage{algorithmicx}
\usepackage{algpseudocode}
\usepackage{url}
\renewcommand{\algorithmicforall}{\textbf{for each}}
\let\ForEach\ForAll

\title{PHASTA Neighbor and Shape Improvement}

\author{Gerrett Diamond, Cameron W. Smith}

\date{\today}

\begin{document}
\maketitle

\textbf{Abstract}. 
Massively parallelized mesh simulations require the mesh to be distributed
throughout many processes(parts) and be balanced in terms of computation and
communicaion. 
Many algorithms exist that target having an equal number mesh entities of a
specified dimension on each part. 
The most powerful of these, in terms of partition quality and runtime are the
multi-level (hyper)graph-based methods and the recusive geometric sectioning
methods. 
Multi-level (hyper)graph methods describe the mesh in terms of graph nodes and
(hyper)edges and then use a v-cycle of coarsening followed by un-coarsening to
divide the graph into subgraphs while recursive sectioning geometric methods use
a coordinate system such as the cartesian location or centroids to recursivly
section the mesh along coordinate or inertial axis. 
The faster geometric methods tend to create parts that have large inter-part
surface area, and many neighbors relative to the (hyper)graph based methods
which causes an increase in communication.
(Hyper)Graph-based methods can also suffer from similar issues as the number of
parts gets very large and the number of elements per part drops to several
hundreds. 
An alternative method that uses mesh data directly, ParMA, performs load 
balancing without the need to construct a graph. 
With access to all the mesh, partition, and model data, including topological and geometric coordinates, partitioning methods analogous to those used by (hyper)graph and geometric based procedures can be defined.
Towards this, ParMA paritioning methods will be extended to improve the overall shape of a part in order to reduce the number of neighbors each part has and thus reduce application communication times.  
 
\section{Introduction}
%Explain previous work in detail with references. 

\section{Methods}
%Tie into controllingUnstructed... for metrics 
%Talk about any math and graph theory


\section{Implementation}
%Explain shape improvement implementation
%List out and refer to algorithms used 

\section{Results}
%Lots of graphs and testing for PHASTA and comparisons to old methods.

\newpage
\bibliographystyle{plain}
\bibliography{references.bib}


\end{document}
