\section{Example Programs}

This chapter presents example programs with PUMI API's.

\subsection{\textit{Model/Mesh loading}}

\begin{small}
\begin{verbatim}
// [INPUT]
// argv[1]: geometric model file
// argv[2]: mesh file
// argv[3]: # parts in mesh file
//
// 1. display system information, # processes running
// 2. load geometric model and mesh 
// 3. display the elapsted time and heap memory increase

#include <mpi.h>
#include <pumi.h>
#include <pumi_errorcode.h>

const char* modelFile = 0;
const char* meshFile = 0;
int num_in_part = 0;

void getConfig(int argc, char** argv)
{
  if ( argc < 4 ) {
    if ( !pumi_rank() )
      printf("Usage: %s <model> <mesh> <num_part_in_mesh>\n", argv[0]);
    MPI_Finalize();
    exit(EXIT_FAILURE);
  }

  modelFile = argv[1];
  meshFile = argv[2];
  num_in_part = atoi(argv[3]);
}

int main(int argc, char** argv)
{
  MPI_Init(&argc,&argv);
  pumi_start();
  pumi_printSys();
  cout<<"Running Test on "<<pumi_size()<<" processes\n";

  // read input args - in-model-file in-mesh-file num-in-part
  getConfig(argc,argv);

  double begin_mem = pumi_getMem(), begin_time=pumi_getTime();

  pGeom g = pumi_geom_load(modelFile);
  pMesh m = pumi_mesh_load(g, meshFile, num_in_part); 

  // print elapsed time and increased heap memory
  pumi_printTimeMem("elapsed time and increased heap memory:", 
             pumi_getTime()-begin_time, pumi_getMem()-begin_mem);

  // clean-up
  pumi_mesh_delete(m);
  pumi_finalize();
  MPI_Finalize();
  
  return PUMI_SUCCESS;
}
\end{verbatim}
\end{small}
%\vspace{-.5cm}\hspace{1cm}

\subsection{\textit{Communication with PCU.h}}

\begin{small}
\begin{verbatim}
// 1. iterate over mesh faces
// 2. if mesh face is ghosted, send the first ghost copy and the mesh face to its first ghost part

#include <mpi.h>
#include <PCU.h>
#include <pumi.h>
#include <pumi_errorcode.h>

int main(int argc, char** argv)
{
  char message[50];
  MPI_Init(&argc,&argv);
  pumi_start();

  ...
  // begin message exchange
  PCU_Comm_Begin();

  pMeshEnt e;
  pMeshIter it = m->begin(2); 
  while ((e = m->iterate(it))) // iterate mesh faces
  {
    if (!pumi_ment_isGhosted(e)) continue; // skip if not ghosted
    Copies temp;
    pumi_ment_getAllGhost(e,temp);
    int to = temp.begin()->first;
    PCU_COMM_PACK(to,temp.begin()->second); // sender
    PCU_COMM_PACK(to,e);
  }
  PCU_Comm_Send();
  while (PCU_Comm_Receive())
  {
    int from = PCU_Comm_Sender();
    pMeshEnt g;
    PCU_COMM_UNPACK(g);
    pMeshEnt s;
    PCU_COMM_UNPACK(s);
  }

  pumi_finalize();
  MPI_Finalize();

  return PUMI_SUCCESS;
}
\end{verbatim}
\end{small}
%\vspace{-.5cm}\hspace{1cm}

\subsection{\textit{Tagging with geometric model}}
\begin{small}
\begin{verbatim}
#include <pumi.h>
#include <string.h>
#include <cstring>

template <class T>
void TEST_TAG (pTag tag, char* in_name, int name_len, int in_type, int in_size)
{
  const char* tag_name;
  // verifying byte tag info
  pumi_tag_getName (tag, &tag_name);
  int tag_type= pumi_tag_getType (tag);
  int tag_size = pumi_tag_getSize (tag);
  int tag_byte= pumi_tag_getByte (tag);
  assert(!strncmp(tag_name, in_name, name_len));
  assert(tag_type == in_type);
  assert(tag_size == in_size);
  assert(tag_byte==sizeof(T)*tag_size);
  assert(!strcmp(tag_name, in_name)
         && tag_type == in_type && tag_size == in_size 
         && ((size_t)tag_byte)==sizeof(T)*tag_size);
}

void TEST_GENT_SETGET_TAG (pGeom g, pGeomEnt ent)
{
  char data[] = "abcdefg";

  pTag pointer_tag=pumi_geom_findTag(g, "pointer");
  pTag int_tag=pumi_geom_findTag(g, "integer");
  pTag long_tag = pumi_geom_findTag(g, "long");
  pTag dbl_tag = pumi_geom_findTag(g, "double");
  pTag ent_tag = pumi_geom_findTag(g, "entity");
  pTag intarr_tag = pumi_geom_findTag(g, "integer array");
  pTag longarr_tag = pumi_geom_findTag(g, "integer array");
  pTag dblarr_tag = pumi_geom_findTag(g, "double array");
  pTag entarr_tag = pumi_geom_findTag(g, "entity array");

  // get the first geometric vertex to use as tag data
  pGeomEnt ent_tag_data=*(g->begin(0));

  // pumi_gent_set/getPtrTag 
  pumi_gent_setPtrTag (ent, pointer_tag, (void*)(data));
  void* void_data = (void*)calloc(strlen(data), sizeof(char));
  pumi_gent_getPtrTag (ent, pointer_tag, &void_data);
  assert(!strcmp((char*)void_data, data));

  // pumi_gent_set/getIntTag 
  pumi_gent_setIntTag(ent, int_tag, 1000);
  int int_data;
  pumi_gent_getIntTag (ent, int_tag, &int_data);
  assert(int_data == 1000);

  // pumi_gent_set/getLongTag
  pumi_gent_setLongTag(ent, long_tag, 3000);
  long long_data; 
  pumi_gent_getLongTag (ent, long_tag, &long_data);
  assert(long_data==3000);

  // pumi_gent_set/getDblTag
  pumi_gent_setDblTag (ent, dbl_tag, 1000.37);
  double dbl_data;
  pumi_gent_getDblTag (ent, dbl_tag, &dbl_data);
  assert(dbl_data == 1000.37);

  // pumi_gent_set/getEntTag
  pumi_gent_setEntTag (ent, ent_tag, ent_tag_data);
  pGeomEnt ent_data; 
  pumi_gent_getEntTag (ent, ent_tag, &ent_data);
  assert(ent_data == ent_tag_data);


 // pumi_gent_set/GetIntArrTag with integer arr tag
  int int_arr[] = {4,8,12};
  int arr_size;
  pumi_gent_setIntArrTag (ent, intarr_tag, int_arr);
  int* int_arr_back = new int[4];
  pumi_gent_getIntArrTag (ent, intarr_tag, &int_arr_back, &arr_size);
  assert(arr_size==3 && int_arr_back[0] == 4 && int_arr_back[1] == 8 && int_arr_back[2] == 12);

   // pumi_gent_set/getDblArrTag
  double dbl_arr[] = {4.1,8.2,12.3};
  pumi_gent_setDblArrTag (ent, dblarr_tag, dbl_arr);
  double* dbl_arr_back = new double[4];
  pumi_gent_getDblArrTag (ent, dblarr_tag, &dbl_arr_back, &arr_size);
  assert(arr_size==3 && dbl_arr_back[0] == 4.1 && dbl_arr_back[1] == 8.2 && 
         dbl_arr_back[2] == 12.3);

  // pumi_gent_set/getEntArrTag
  pGeomEnt* ent_arr = new pGeomEnt[3];
  ent_arr[0] = ent_arr[1] = ent_arr[2] = ent_tag_data;
  pumi_gent_setEntArrTag (ent, entarr_tag, ent_arr);
  pGeomEnt* ent_arr_back = new pGeomEnt[4];
  pumi_gent_getEntArrTag (ent, entarr_tag, &ent_arr_back, &arr_size);
  assert(arr_size==3 && ent_arr_back[0] == ent_tag_data && ent_arr_back[1] == 
           ent_tag_data && ent_arr_back[2] == ent_tag_data
           && ent_arr[0]==ent_arr_back[0] && ent_arr[1]==ent_arr_back[1] && 
           ent_arr[2] == ent_arr_back[2]);

  delete [] int_arr_back;
  delete [] long_arr_back;
  delete [] dbl_arr_back;
  delete [] ent_arr;
  delete [] ent_arr_back;
}

void TEST_GENT_DEL_TAG (pGeom g, pGeomEnt ent)
{
  pTag pointer_tag=pumi_geom_findTag(g, "pointer");
  pTag int_tag=pumi_geom_findTag(g, "integer");
  pTag long_tag = pumi_geom_findTag(g, "long");
  pTag dbl_tag = pumi_geom_findTag(g, "double");
  pTag ent_tag = pumi_geom_findTag(g, "entity");
  pTag intarr_tag = pumi_geom_findTag(g, "integer array");
  pTag longarr_tag = pumi_geom_findTag(g, "long array");
  pTag dblarr_tag = pumi_geom_findTag(g, "double array");
  pTag entarr_tag = pumi_geom_findTag(g, "entity array");

  pumi_gent_deleteTag(ent, pointer_tag);
  pumi_gent_deleteTag(ent, int_tag);
  pumi_gent_deleteTag(ent, long_tag);
  pumi_gent_deleteTag(ent, dbl_tag);
  pumi_gent_deleteTag(ent, ent_tag);
  pumi_gent_deleteTag(ent, intarr_tag);
  pumi_gent_deleteTag(ent, longarr_tag);
  pumi_gent_deleteTag(ent, dblarr_tag);
  pumi_gent_deleteTag(ent, entarr_tag);

  assert(!pumi_gent_hasTag(ent, pointer_tag));
  assert(!pumi_gent_hasTag(ent, int_tag));
  assert(!pumi_gent_hasTag(ent, long_tag));
  assert(!pumi_gent_hasTag(ent, dbl_tag));
  assert(!pumi_gent_hasTag(ent, ent_tag));
  assert(!pumi_gent_hasTag(ent, intarr_tag));
  assert(!pumi_gent_hasTag(ent, longarr_tag));
  assert(!pumi_gent_hasTag(ent, dblarr_tag));
  assert(!pumi_gent_hasTag(ent, entarr_tag));
}

void TEST_GEOM_TAG(pGeom g)
{
  pTag pointer_tag = pumi_geom_createTag(g, "pointer", PUMI_PTR, 1);
  pTag int_tag = pumi_geom_createTag(g, "integer", PUMI_INT, 1);
  pTag long_tag = pumi_geom_createTag(g, "long", PUMI_LONG, 1);
  pTag dbl_tag = pumi_geom_createTag(g, "double", PUMI_DBL, 1);
  pTag ent_tag = pumi_geom_createTag(g, "entity", PUMI_ENT, 1);

  pTag intarr_tag=pumi_geom_createTag(g, "integer array", PUMI_INT, 3);
  pTag longarr_tag=pumi_geom_createTag(g, "long array", PUMI_LONG, 3);
  pTag dblarr_tag = pumi_geom_createTag(g, "double array", PUMI_DBL, 3);
  pTag entarr_tag = pumi_geom_createTag(g, "entity array", PUMI_ENT, 3);

  // verifying tag info
  TEST_TAG<void*>(pointer_tag, "pointer", strlen("pointer"), PUMI_PTR, 1);
  TEST_TAG<int>(int_tag, "integer", strlen("integer"), PUMI_INT, 1);
  TEST_TAG<long>(long_tag, "long", strlen("long"), PUMI_LONG, 1);
  TEST_TAG<double>(dbl_tag, "double", strlen("double"), PUMI_DBL, 1);
  TEST_TAG<pMeshEnt>(ent_tag, "entity", strlen("entity"), PUMI_ENT, 1);

  TEST_TAG<int>(intarr_tag, "integer array", strlen("integer array"), PUMI_INT, 3);
  TEST_TAG<long>(longarr_tag, "long array", strlen("long array"), PUMI_LONG, 3);
  TEST_TAG<double>(dblarr_tag, "double array", strlen("double array"), PUMI_DBL, 3);
  TEST_TAG<pMeshEnt>(entarr_tag, "entity array", strlen("entity array"), PUMI_ENT, 3);

  assert(pumi_geom_hasTag(g, int_tag));
  pTag cloneTag = pumi_geom_findTag(g, "pointer");
  assert(cloneTag);
  std::vector<pTag> tags;
  pumi_geom_getTag(g, tags);
  assert(cloneTag == pointer_tag && tags.size()==9);

  for (gGeomIter gent_it = g->begin(0); gent_it!=g->end(0);++gent_it)
  {
    TEST_GENT_SETGET_TAG(g, *gent_it);
    TEST_GENT_DEL_TAG(g, *gent_it);
  }

  // delete tags
  for (std::vector<pTag>::iterator tag_it=tags.begin(); tag_it!=tags.end(); ++tag_it)
    pumi_geom_deleteTag(g, *tag_it);

  tags.clear();
  pumi_geom_getTag(g, tags);

  assert(!tags.size());
}

int main(int argc, char** argv)
{
  MPI_Init(&argc,&argv);
  pumi_start();
  ...
  getConfig(argc,argv);
  pGeom g = pumi_geom_load(modelFile);
  TEST_GEOM_TAG(g);
  ...  
  return PUMI_SUCCESS;
}
\end{verbatim}
\end{small}
%\vspace{-.5cm}\hspace{1cm}

\subsection{Mesh/Entity information}
\begin{small}
\begin{verbatim}
int main(int argc, char** argv)
{
  ...
  pMeshEnt e;
  vector<pMeshEnt> adj_ents;
  int mesh_dim=pumi_mesh_getDim(m);
  pMeshIter mit = m->begin(mesh_dim);
  while ((e = m->iterate(mit)))
  {
    assert(pumi_ment_getDim(e)==mesh_dim);
    // check # one-level up adjancent entities
    assert(pumi_ment_getNumAdj(e, mesh_dim+1)==0);
    // check # one-level down adjancent entities
    adj_ents.clear();
    pumi_ment_getAdj(e, mesh_dim-1, adj_ents);
    assert(adj_ents.size()==pumi_ment_getNumAdj (e, mesh_dim-1));
    if (!pumi_ment_isOnBdry(e)) continue; // skip internal entity
    // if entity is on part boundary, count remote copies    
    Copies copies;
    pumi_ment_getAllRmt(e,copies);
    // check #remotes
    assert (pumi_ment_getNumRmt(e)==copies.size() && copies.size()>0);
    // check the entity is not ghost or ghosted
    assert(!pumi_ment_isGhost(e) && !pumi_ment_isGhosted(e));
  }
  m->end(mit);
  ...
}
\end{verbatim}
\end{small}
%\vspace{-.5cm}\hspace{1cm}


\subsection{2D mesh construction}
\begin{small}
\begin{verbatim}
  // create an empty model and 2D mesh
  pGeom g = pumi_geom_load (NULL, "null");
  pMesh m = pumi_mesh_create(g, 2);

  double xyz[3];
  std::vector<pMeshEnt> new_vertices;
  std::vector<pMeshEnt> new_edges;
  pMeshEnt vertices[2];
  pMeshEnt edges[3];

  for (int i=0; i<num_new_vertices; ++i)
  {
    //create a new vertex and store in vector "new_vertices"
    new_vertices.push_back (pumi_mesh_createVtx(m, NULL, xyz));
  }

  for (size_t i=0; i< new_vertices.size()/2-1; ++i)
  {
    vertices[0] = new_vertices[i];    
    vertices[1] = new_vertices[i+1]; 
    // create a new edge and store in vector "new_edges"
    new_edges.push_back(pumi_mesh_createEnt(m, NULL, 1, vertices));
  }

  // note face-edge order is ignored in this example
  for (size_t i=0; i< new_edges.size()/3-1; ++i)
  {
    for (int k=0; k<3; ++k)
      edges[k] = new_edges[i+k];     
    pumi_mesh_createEnt(m, NULL, 2, edges);
  }

  // finalize mesh entity creation
  pumi_mesh_freeze(m);
\end{verbatim}
\end{small}

\subsection{\textit{Mesh tagging}}
\begin{small}
\begin{verbatim}
#include "pumi.h"

int main(int argc, char** argv)
{
  ...
  pMesh m;
  pMeshEnt e;
  ...
  pMeshTag int_tag = pumi_mesh_findTag(m, "integer");
  if (!int_tag) // int_tag doesn't exist
    int_tag = pumi_mesh_createIntTag(m, "integer", 1);
  pMeshTag long_tag = pumi_mesh_createLongTag(m, "long", 1);
  pMeshTag dbl_tag = pumi_mesh_createDblTag(m, "double", 1);

  pMeshTag intarr_tag=pumi_mesh_createIntTag(m, "integer array", 3);
  pMeshTag longarr_tag=pumi_mesh_createLongTag(m, "long array", 3);
  pMeshTag dblarr_tag = pumi_mesh_createDblTag(m, "double array", 3);


  // set/get integer tag
  int int_value=pumi_ment_getLocalID(ent), int_data;
  pumi_ment_setIntTag(ent, int_tag, &int_value);
  pumi_ment_getIntTag (ent, int_tag, &int_data);
  assert(int_data == int_value);
  ...
  // set/get double array tag
  double dbl_arr[] = {4.1,8.2,12.3};
  pumi_ment_setDblTag (ent, dblarr_tag, dbl_arr);
  double* dbl_arr_back = new double[3];
  pumi_ment_getDblTag (ent, dblarr_tag, dbl_arr_back);
  ...
  // retrieve all tags on mesh
  std::vector<pMeshTag> tags;
  pumi_mesh_getTag(m, tags);
  ...
  // delete tags
  for (std::vector<pMeshTag>::iterator tag_it=tags.begin(); tag_it!=tags.end(); ++tag_it)
    pumi_mesh_deleteTag(m, *tag_it);
  ...
}\end{verbatim}
\end{small}

\subsection{Mesh partitioning}
\begin{small}
\begin{verbatim}
// 1. load 2D serial mesh
// 2. create a migration plan to partition the mesh as per model faces.
//    (if mesh face is classified on model face i (i>0), send the mesh face to part i.
// 3. check mesh validity

Migration* get_plan_per_model(pGeom g, pMesh m)
{
  int dim = pumi_mesh_getDim(m);
  int num_peers = pumi_size();
  Migration* plan = new Migration(m);
  if (!pumi_rank()) return plan; // only master process construct the plan

  pMeshEnt e;
  int num_gface = pumi_geom_getNumEnt(g, dim);
  assert(num_gface==pumi_size());
  int gface_id;
  int dest_pid;
  pMeshIter it = m->begin(2); // face
  while ((e = m->iterate(it))) 
  { 
    pGeomEnt gface = pumi_ment_getGeomClas(e); // get the classification
    gface_id = pumi_gent_getID(gface); // get the geom face id
    dest_pid = gface_id-1;
    plan->send(e, dest_pid);    
  }
  m->end(it);
  return plan;
}

int main(int argc, char** argv)
{
  ...
  pGeom g = pumi_geom_load(modelFile);
  pMesh m = pumi_mesh_loadSerial(g, meshFile);
  Migration* plan = get_plan_per_model(g, m);
  pumi_mesh_migrate(m, plan);
  pumi_mesh_verify(m);
  ...
}
\end{verbatim}
\end{small}

\subsection{Mesh distribution}
\begin{small}
\begin{verbatim}
// 1. load 2D serial mesh
// 2. send 1/5 of mesh elements to three parts i-1, i, i+i.
//    i is modulo(element's local ID,# processes)
// 3. write the distributed mesh into file "part.smb"

int main(int argc, char** argv)
{
  ...
  pGeom g = pumi_geom_load(modelFile);
  pMesh m = pumi_mesh_loadSerial(g, meshFile);

  Distribution* plan = new Distribution(m);

  int dim=pumi_mesh_getDim(m), count=0, pid;
  pMeshIter it = m->begin(dim);
  while ((e = m->iterate(it)))
  {
    pid=pumi_ment_getLocalID(e)%pumi_size();
    plan->send(e, pid);
    if (pid-1>=0) plan->send(e, pid-1);
    if (pid+1<pumi_size()) plan->send(e, pid+1);
    if (count==pumi_mesh_getNumEnt(m, dim)/5) break;
    ++count;
  }
  m->end(it);

  pumi_mesh_distribute(m, plan);

  // write mesh in part.smb
  pumi_mesh_write(m,"part.smb"); 
}
\end{verbatim}
\end{small}

\subsection{Computing the area of adjacent faces}
\begin{small}
\begin{verbatim}
// loop over mesh vertices and calculate the area of adjacent faces
// if a mesh vertex is on part boundary, do communications to add 
// the area of remote faces
  pMeshEnt e;
  pMeshTag area_tag = pumi_mesh_createDblTag(m, "area", 1);

  PCU_Comm_Begin();  // start communication

  pMeshIter it = m->begin(0);
  while ((e = m->iterate(it)))
  {
    double area = 0.;
    std::vector<pMeshEnt> faces;
    pumi_ment_getAdj(e, 2, faces);
    for (std::vector<pMeshEnt>::iterator fit=faces.begin(); fit!=faces.end(); ++fit)
      area += get_face_area(m, *fit); // get_face_area computes the face area
    pumi_ment_setDblTag(e, area_tag, &area);
    if (!pumi_ment_isOnBdry(e)) continue;
    Copies remotes;
    pumi_ment_getAllRmt(e,remotes);
    // loop over remote copies and send message to remote copies
    for (pCopyIter rit = remotes.begin(); rit != remotes.end(); ++rit)
    {
      PCU_COMM_PACK(rit->first, rit->second);
      PCU_Comm_Pack(rit->first, &area, sizeof(double));
    }
  }
  m->end(it);
  PCU_Comm_Send(); // send messages collected in buffers

  // part boundary vertices receive area from the remote copies and add them up
  while (PCU_Comm_Listen())
    while (!PCU_Comm_Unpacked())
    {
      pMeshEnt rmt_e;
      PCU_COMM_UNPACK(rmt_e);
      double rmt_area;
      PCU_Comm_Unpack(&rmt_area, sizeof(double));
      double area;
      pumi_ment_getDblTag (rmt_e, area_tag, &area);
      area+=rmt_area;
      pumi_ment_setDblTag (rmt_e, area_tag, &area);      
    }
\end{verbatim}
\end{small}


\subsection{Entity-wise ghosting}
\begin{small}
\begin{verbatim}
Ghosting* getGhostingPlan(pMesh m)
{
  int mesh_dim=pumi_mesh_getDim(m);
  pMeshEnt e;
  Ghosting* plan = new Ghosting(m, mesh_dim);
  {
    pMeshIter it = m->begin(mesh_dim);
    int count=0, pid;
    while ((e = m->iterate(it)))
    {
      for (int i=0; i<pumi_size()/2; ++i)
      {
        pid = (pumi_ment_getGlobalID(e)+rand())%pumi_size();
        plan->send(e, pid);
      }
      ++count; 
     if (count==pumi_mesh_getNumEnt(m, mesh_dim)/3) break;
    }
    m->end(it);
  }
  return plan;
}

int main(int argc, char** argv)
{
  ...
  int mesh_dim=pumi_mesh_getDim(m);
  pMeshEnt e;

  int* org_mcount=new int[4];
  for (int i=0; i<4; ++i)
    org_mcount[i] = pumi_mesh_getNumEnt(m, i);

  Ghosting* ghosting_plan = get_ghosting_plan(m);
  pumi_ghost_create(m, ghosting_plan);

  int num_ghost_vtx=0;
  pMeshIter mit = m->begin(0);
  while ((e = m->iterate(mit)))
  {
    if (pumi_ment_isGhost(e))
    {
      ++num_ghost_vtx;
     assert(pumi_ment_getOwnPID(e)!=pumi_rank());
    }
  }   
  m->end(mit);
  assert(num_ghost_vtx+org_mcount[0]==pumi_mesh_getNumEnt(m,0));
  pumi_mesh_verify(m); // this should throw an error message
  pumi_ghost_delete(m);
  for (int i=0; i<4; ++i)
    assert(org_mcount[i] == pumi_mesh_getNumEnt(m, i));
  ...
}
\end{verbatim}
\end{small}

\subsection{ Layer-wise ghosting}
\begin{small}
\begin{verbatim}
int main(int argc, char** argv)
{
  ...
  int mesh_dim=pumi_mesh_getDim(m);

  int* org_mcount=new int[4];
  for (int i=0; i<4; ++i)
    org_mcount[i] = pumi_mesh_getNumEnt(m, i);

  for (int brg_dim=mesh_dim-1; brg_dim>=0; --brg_dim)
    for (int num_layer=1; num_layer<=3; ++num_layer)
      for (int include_copy=0; include_copy<=1; ++include_copy)
      {
        pumi_ghost_createLayer (m, brg_dim, mesh_dim, num_layer, include_copy);
        pumi_ghost_delete(m);
        pumi_mesh_verify(m);
        for (int i=0; i<4; ++i)
          assert(org_mcount[i] == pumi_mesh_getNumEnt(m, i));
      }
  ...
}
\end{verbatim}
\end{small}

\subsection{ Accumulative layer ghosting}
\begin{small}
\begin{verbatim}
int main(int argc, char** argv)
{
  ...
  int mesh_dim=pumi_mesh_getDim(m);

  int* org_mcount=new int[4];
  for (int i=0; i<4; ++i)
    org_mcount[i] = pumi_mesh_getNumEnt(m, i);

  for (int brg_dim=mesh_dim-1; brg_dim>=0; --brg_dim)
    for (int num_layer=1; num_layer<=3; ++num_layer)
      for (int include_copy=0; include_copy<=1; ++include_copy)
        pumi_ghost_createLayer (m, brg_dim, mesh_dim, num_layer, include_copy);

  pumi_ghost_delete(m);
  pumi_mesh_verify(m);
  for (int i=0; i<4; ++i)
    assert(org_mcount[i] == pumi_mesh_getNumEnt(m, i));
  ...
}
\end{verbatim}
\end{small}

\subsection{ Field shape}
\begin{small}
\begin{verbatim}
  // get the default shape of the mesh - only vertex has a node
  pShape s = pumi_mesh_getShape(m);
  assert(pumi_shape_getNumNode(s, 1)==0);

  // change shape to Lagrange with order 2 - edge has a node
  pumi_mesh_setShape(m, pumi_shape_getLagrange(2));
  assert(pumi_shape_getNumNode(pumi_mesh_getShape(m), 1)==1);

  double xyz[3];
  pMeshIter it = m->begin(1);
  pMeshEnt e;
  // loop over edges and get and set the node coordinates
  while ((e = m->iterate(it)))
  {
    pumi_node_getCoord(e, 0, xyz);
    for (int i=0; i<3; ++i) xyz[i] += 0.5;
    pumi_node_setCoord(e, 0, xyz);
  }
  m->end(it);
\end{verbatim}
\end{small}

\subsection{ Field manipulation}
\begin{small}
\begin{verbatim}
  int num_dofs_per_node=3;
  pField f =pumi_field_create(m, "xyz_field", num_dofs_per_node);
  assert(pumi_field_getName(f)==std::string("xyz_field"));
  assert(pumi_field_getType(f)==PUMI_PACKED);
  assert(pumi_field_getSize(f)==num_dofs_per_node);

  // create global numbering for field node
  pGlobalNumbering gn = pumi_mesh_createNumbering(m, "xyz_numbering", pumi_field_getShape(f));

  // fill the dof data
  double xyz[3];
  pMeshIter it = m->begin(0);
  pMeshEnt e;
  while ((e = m->iterate(it)))
  {
    if (!pumi_ment_isOwned(e)) continue;
    pumi_node_getCoord(e, 0, xyz);
    if (pumi_ment_isOnBdry(e)) 
      for (int i=0; i<3;++i) 
        xyz[i] *= pumi_ment_getLocalID(e);
    pumi_ment_setField(e, f, 0, xyz);
  }
  m->end(it);

  pumi_field_synchronize(f); // synchronize field value between remote copies

  double data[3];
  it = m->begin(0);
  while ((e = m->iterate(it)))
  {
    if (!pumi_ment_isOwned(e)) continue;
    pumi_node_getCoord(e, 0, xyz);
    pumi_ment_getField(e, f, 0, data);
    for (int i=0; i<3;++i) 
      if (pumi_ment_isOnBdry(e)) 
        assert(data[i] == xyz[i]*pumi_ment_getLocalID(e));
  }
  m->end(it);

  // delete global numbering
  pumi_mesh_deleteNumbering(gn);

\end{verbatim}
\end{small}
