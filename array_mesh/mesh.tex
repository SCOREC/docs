\documentclass{article}
\title{Efficient Data Structures for Topological Meshes}
\author{Dan Ibanez \and Mark S Shephard}

\begin{document}
\maketitle
\begin{abstract}
Topological data structures are useful in many areas.
The BRep family of data structures for CAD models
are topological complexes,
as are the various mesh data structures used in
finite element applications.
We present a graph-theoretic foundation for
these data structures and derive an efficient
finite element
mesh data structure based on several optimizations
to a generic graph data structure.
The result is a compact array-based structure that is also
easy to modify locally at runtime
\end{abstract}

\section{Introduction}
Describe the needs of parallel adaptive mesh data structures

\section{Related Work}
STK, MOAB, FMDB, etc...

\section{Graph Data Structure}
Describe an object-oriented graph data structure that allows
constant-time addition and removal of edges and vertices.

\begin{verbatim}
struct vertex {
  struct edge* first_edge_from;
  struct edge* first_edge_to;
};

struct edge {
  struct vertex* from;
  struct edge* next_edge_from;
  struct edge* prev_edge_from;
  struct edge* next_edge_to;
  struct edge* prev_edge_to;
  struct vertex* to;
};
\end{verbatim}

\section{Structure of Arrays}
Describe how object-oriented, pointer-based, linked structures
can be recast into arrays of indices to themselves.
Also include here the algorithms for object addition and removal.

\section{Topology Graph}
Reduce the representation of a topological complex
to the topological adjacency graph.

Include several examples of mesh topology graphs,
relate to the various mesh representations.

\section{Mesh Structure}
Derive the mesh structure by starting with a
one-level topology graph and applying optimizations
including recasting into arrays.

% \bibliographystyle{plain}
% \bibliography{topo}

\end{document}
