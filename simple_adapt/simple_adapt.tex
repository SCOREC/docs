\documentclass{article}
\usepackage{amsmath}
\usepackage{tikz}
\usepackage{titling}

\setlength{\droptitle}{-10em}

\title{A Simple Description of Adapt Operators}
\author{Dan Ibanez}

\begin{document}
\maketitle

This document describes in simple terms three key
mesh modification operators, from the perspective
of solution transfer.
We illustrate these operators in 2D because
the majority of transfer concerns are visible in
both 2D and 3D.
The edge split, edge collapse, and edge swap
operators are chosen because they are the bare minimum
required to carry out all parts of adaptivity.
In practice, more operators may be used.

Each operator can be viewed as acting on a group
of elements called a cavity, and replacing that
cavity with a different one.
We explain how transfer is done for linear fields
but don't go into much detail for other fields.
Transfer algorithms are likely to depend on the
physical meaning of a field and the
chosen shape functions (``element type"),
though some general algorithms have been proposed.
More information can be added to this document as
readers provide comments.

\section{Edge Split}

{\center\includegraphics[width=0.6\textwidth]{edge_split.png}}

This is the simplest ``refine" operator.
A new vertex is placed at the midpoint of
and existing edge; all elements adjacent to
that edge are split into two elements.

The first concern would be a degree of freedom
at the new vertex.
Typically we interpolate its value along the
edge based on the values at the endpoints,
assuming the elements are interpolating and
at least $C^{0}$ continuous.
This transfer is exact in the sense that the
field value at all points inside the new cavity is exactly the
same as in the old cavity.

If there is a $C^{-1}$ field over the elements,
that too can be transferred exactly by
having the child elements inherit the value
from the parents.

Edge and 3D face degrees of freedom can still
be interpolated from the parent elements,
but care should be taken to derive the results
depending on the shape functions used.

\section{Edge Collapse}

{\center\includegraphics[width=0.6\textwidth]{edge_collapse.png}}

This is the simplest ``coarsen" or ``de-refine" operator.
An edge and its adjacent elements are removed
by moving one of its endpoints onto the other.

By definition, this is a coarser representation and
so some accuracy will be lost; the transfer cannot
be exact.
However, one may still preserve certain quantities.

In the case linear shape functions, we typically
don't do anything in terms of transfer; we leave
the remaining vertex values as they are.
It is important that the adapt operators not modify
the boundary of the cavity, including field values.

When designing a transfer algorithm for coarsening,
it may be useful to remember that every new entity
in the cavity (element, face, or edge) corresponds
to one of the old entities, via the transformation
for moving the central vertex towards one of the
boundary vertices.
The internal edges and triangles in the 2D figure illustrate this.

\section{Edge Swap}

{\center\includegraphics[width=0.6\textwidth]{edge_swap.png}}

This is one of the most effective ``shape correction" operators.
The cavity around an edge is reconnected such that
that edge is no longer there.
In 2D that leaves only one other choice, but in 3D
there are many resulting choices.

This operator does not add or remove vertices, so for linear
fields we again don't do any special transfer.

For any other field, the challenge in edge swapping is that
the new entities don't have a clear-cut relationship
to the old entities as they did in splitting/collapsing,
so the cavity should be considered
as a whole when designing a transfer algorithm.

\end{document}
